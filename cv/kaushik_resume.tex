%%%%%%%%%%%%%%%%%%%%%%%%%%%%%%%%%%%%%%%%%
% Medium Length Professional CV
% LaTeX Template
% Version 2.0 (8/5/13)
%
% This template has been downloaded from:
% http://www.LaTeXTemplates.com
%
% Original author:
% Rishi Shah 
%
% Important note:
% This template requires the resume.cls file to be in the same directory as the
% .tex file. The resume.cls file provides the resume style used for structuring the
% document.
%
%%%%%%%%%%%%%%%%%%%%%%%%%%%%%%%%%%%%%%%%%

%----------------------------------------------------------------------------------------
%    PACKAGES AND OTHER DOCUMENT CONFIGURATIONS
%----------------------------------------------------------------------------------------

\documentclass{resume} % Use the custom resume.cls style

\usepackage[left=0.75in,top=0.5in,right=0.75in,bottom=0.6in ]{geometry} % Document margins
\usepackage{graphicx}
\graphicspath{ {.} }
\usepackage{wrapfig}
\usepackage{float}
% \usepackage{url}
\usepackage{fontawesome}
\usepackage{url} 
% \usepackage{hyperref}

\newcommand{\tab}[1]{\hspace{.2667\textwidth}\rlap{#1}}
\newcommand{\itab}[1]{\hspace{0em}\rlap{#1}}


\begin{document}

\begin{minipage}{0.8\textwidth}

{\Huge  \bf Kaushik Amar Das}\vspace{10px} \\
% {\fontsize{14}{14}{\faMobile}} \enspace +91 None
\par
\faGlobe \enspace \url{https://cozek.github.io/} \par
\faEnvelope \enspace kaushikamardas@gmail.com \par
\faGithubSquare \enspace \url{www.github.com/cozek} \par
\faLinkedinSquare \enspace \url{www.linkedin.com/in/kaushikamardas} \par
\end{minipage}
\noindent\begin{minipage}{0.2\textwidth}% adapt widths of minipages to your needs
% \includegraphics[ scale = 0.5]{iiitg.png}
\end{minipage}%
\hfill%\\
\newline




%----------------------------------------------------------------------------------------
%    EDUCATION SECTION
%----------------------------------------------------------------------------------------


\begin{rSection}{Education}  

    {\bf \large MTech in Computer Science and Enginnering}
    \hfill \textit{CGPA: 9.60 / 10}
    \\ \textit{Indian Institute of Information Technology, Guwahati} 
    \hfill {\em July 2018 - June 2020}
    \par\vspace{7px}

    {\bf \large BTech in Computer Science and Enginnering}
    \hfill \textit{CGPA: 7.04 / 10}
    \\ \textit{Assam Science and Technology University, India} 
    \hfill {\em July 2014 - June 2018}
    \par\vspace{7px}


% \begin{table}[ht]
%     \centering
%     \begin{tabular}{lr}
%         \textbf{MTech in Computer Science and Enginnering} \newline & \hfill { CGPA: 9.47 } \\
%         { Indian Institute of Information Technology, Guwahati} & {\em July 2018 - Present} \\
%         % \hline
%         & \\
%         \textbf{BTech in Computer Science and Enginnering} &  { CGPA: 7.04 }  \\
%         { Assam Science and Technology University, India} & {\em July 2014 - June 2018} \\
%         \hline
%         & \\
%         \textbf{Senior School} & { Percentage: 77 }\\
%         { Shrimanta Shankar Academy, Dispur} & {\em  March 2012 - June 2014} \\
%         % \hline
%         & \\
%         \textbf{High School} & { Percentage: 82.5 } \\
%         { St. Stephen's School, Christian Basti} & {\em February 2012}
%     \end{tabular}
%     \label{tab:my_label}
% \end{table}{}

% \textbf{Master of Technology, Computer Science and Enginnering}
%  \hfill { CGPA: 9.47 }
% \\ { Indian Institute of Information Technology, Guwahati} \hfill {\em July 2018 - Present} \par

% \textbf{Bachelor of Technology, Computer Science and Enginnering}
% \hfill { CGPA: 7.04 } 
% \\{ Assam Science and Technology University, Assam} \hfill {\em July 2014 - June 2018} 
% \par

% \textbf{XII} \hfill { Percentage: 77 }
% \\{ Shrimanta Shankar Academy, Dispur} \hfill {\em  March 2012 - June 2014} 
% \par

% \textbf{HSLC} \hfill { Percentage: 82.5 }
% \\{ St. Stephen's School, Christian Basti} \hfill {\em February 2012}

\end{rSection}

%----------------------------------------------------------------------------------------
%   Research Works
%----------------------------------------------------------------------------------------
\newcommand{\somepadding}{\vspace{10px}}
\begin{rSection}{Research Works}
    \label{pub:hasoc}
         {\bf Kaushik Amar Das}, Arup Baruah, Ferdous Ahmed Barbhuiya and Kuntal Dey.
         KAFK at SemEval 2020 Task 8: Extracting Features From Pre-trained Neural Networks To 
         Classify Internet Memes. In \textit{Proceedings of the 14th International Workshop 
         on Semantic Evaluation}, COLING ’20, Barcelona, Spain, December, 2020. 
         Association for Computational Linguistics. 
         \emph{Submitted for publication, currently under review}
         \\Code: \url{https://github.com/cozek/memotion2020-code}
         \somepadding
    
        {\bf Kaushik Amar Das}, Arup Baruah, Ferdous Ahmed Barbhuiya and 
        Kuntal Dey. KAFK at SemEval 2020 Task 12: Checkpoint Ensemble of 
        Transformers for Hate Speech Classification.  In \textit{Proceedings 
        of the 14th International Workshop on Semantic Evaluation},  
        COLING ’20, Barcelona, Spain, December, 2020. Association for 
        Computational Linguistics.
        \emph{Submitted for publication, currently under review}
        \\ Code: \url{https://github.com/cozek/OffensEval2020-code}
        \somepadding


        {\bf Kaushik Amar Das}. Hate Speech Detection in Social Media. 
        Masters Thesis under supervision of Ferdous Ahmed Barbhuiya. 
        Indian Institute of Information Technology, Guwahati, India, June, 2020.
        online \url{https://github.com/cozek/hate-speech-detection-social-media}
        \somepadding


        Arup Baruah, {\bf Kaushik Amar Das}, Ferdous Ahmed Barbhuiya and 
        Kuntal Dey. Aggression Identification in English, Hindi and Bangla 
        Text using BERT, RoBERTa and SVM.  In \textit{Proceedings of the 
        Second Workshop on Trolling, Aggression and Cyberbullying},  
        LREC ’20, pages 76--82,  Marseille, France, May, 2020. European 
        Language Resources Association.  
        online \url{https://www.aclweb.org/anthology/2020.trac-1.12.pdf}
        \\Code: \url{https://github.com/cozek/trac2020_submission}
        \somepadding

        {\bf Kaushik Amar Das}, Ferdous Ahmed Barbhuiya. 
        Team FalsePostive at HASOC 2019: Transfer-Learning for 
        Detection and Classification of Hate Speech. 
        In \textit{Proceedings of the 11th annual meeting of the Forum 
        for Information Retrieval Evaluation}, FIRE'19, Kolkata, India, 
        December, 2019. CEUR-WS.org, 
        online \url{http://ceur-ws.org/Vol-2517/T3-19.pdf}
        \\Code: \url{https://github.com/cozek/hasoc-2019-falsepostive}
        \somepadding
    
    \end{rSection}
% %--------------------------------------------------------------------------------
% %    Masters Thesis
% %-----------------------------------------------------------------------------------------------

% \begin{rSection}{Masters Thesis}
% {\bf Hate Speech Detection in Social Media  }
% \hfill {\em July 2019 - June 2020}
% \\{\em Under the guidance of Dr Ferdous Ahmed Barbhuiya, IIIT Guwahati}
% \\ Designed and evaluated automated systems for detecting hate speech in social media posts using deep neural networks. These systems were built using architectures such as CNN, BiLSTM and transformer-based architectures like GPT-2, RoBERTa, XLM-RoBERTa. In total, the thesis worked on datasets in 4 different languages (English, Hindi, Indian Bengali and German) and 2 different domains, social media text and internet memes. The datasets used were from recent research workshops such as OffenEval'20, Memotion Analysis, TRAC-2 and HASOC'19. 

% Online at: \url{https://github.com/cozek/hate-speech-detection-social-media}


% \end{rSection}

%--------------------------------------------------------------------------------
%    Projects And Seminars
%-----------------------------------------------------------------------------------------------
\begin{rSection}{Achievements}
    Ranked 2\textsuperscript{nd} out of 10 in Misogynistic Aggression Identification subtask in the 
    Hindi language using XLM - RoBERTa in the second workshop on Trolling, Aggression and Cyberbullying - 2020
    
    Ranked 9\textsuperscript{th} out of 29 in Humor Classification task of 
    SemEval 2020 Task 8, Memotion Analysis, using EfficientNet.
\end{rSection}
%--------------------------------------------------------------------------------
%    Projects And Seminars
%-----------------------------------------------------------------------------------------------
\begin{rSection}{Other Projects}

{\bf Smart Garbage Management and Monitoring System}
\hfill {\em March-May 2019}
\\{\em Class Project under Instructor Asst. Prof. Rakesh Matam, IIIT Guwahati}
\\Developed a smart garbage management system that would alert the concerned authorities when the dustbins are full and need emptying and monitor the status of the dustbins in real-time. 
\\{Technology Used: }Python, Raspberry Pi, AWS EC2, AWS SNS, AWS IoT Core \par


{\bf Twitter Sentimental Analysis using Apache Spark and Kafka}
\hfill {\em April - May 2019}
\\{\em Class Project under Instructor Asst. Prof. Dip Sankar Banerjee, IIIT Guwahati}
\\Mined tweets with certain hashtags from Twitter and pipelined it through Apache Kafka to Apache Spark's streaming API and analyzed their sentiment.
\\{Technology Used: }Docker, Apache Spark, Apache Kafka, Apache Toree, Scala \par

{\bf Result Portal}
\hfill {\em Oct - Nov 2018}
\\{\em Class Project under Instructor Prof. Gautam Barua, IIIT Guwahati}
\\Developed an online portal where the semester results of the students would be uploaded and managed.
\\
{ Technology Used: }XAMPP, PHP, MySQL, HTML, CSS, Bootstrap.\par

{\bf Intl. Local Search for solving computationally hard problems}
\hfill {\em Jan - June 2018}
%\\{\bf Solving Computationally Hard Problems}
\\{\em Under the guidance of Asst. Prof. Arnab Kr. Mishra, Royal Global University }
\\Evaluated the performance of various algorithms
for solving computationally hard problems.
\\{ Technology Used: }Python \par


\end{rSection}



%----------------------------------------------------------------------------------------
%    TECHNICAL STRENGTHS SECTION
%----------------------------------------------------------------------------------------

\begin{rSection}{Technical Summary}

    \begin{tabular}{ @{} >{\bfseries}l @{\hspace{6ex}} l }
    Programming  \ & Proficient: Python \\
                       \ & Amateur: C++ \\
                       \ & Previous Experience: Scala, Java, MATLAB, PHP, C \\
    Markdown  \ & HTML, CSS, Bootstrap \\
    Environments \ & Windows, Linux \\
    Software Tools  \ & Jupyter Notebook, VSCode, Docker \\
    Frameworks  \ & Numpy, Pandas, Scikit-Learn, NLTK, Keras, PyTorch, Apache Spark \\
                \ & Apache Hadoop, Apache Hive\\
    \end{tabular}
    
    \end{rSection}

%----------------------------------------------------------------------------------------
%    WORK EXPERIENCE SECTION
%----------------------------------------------------------------------------------------

\begin{rSection}{Internship and Certification}

{\bf \large Deep Learning Specialization}
\hfill \textit{June 2019}
\\ \textit{by deeplearning.ai} on \emph{Coursera}
% \begin{itemize}
%     \item Neural Networks and Deep Learning
%     % \\ Certificate ID: G8JQDASSXEBS 
%     \item Improving Deep Neural Networks
%     % \\ Certificate ID: LXVY282B84HN
%     \item Sequence Models
%     % \\ Certificate ID: SFXDCSRZL5PJ
%     \item Structuring Machine Learning Projects
%     % \\ Certificate ID: D6BL5HP99QR5
% \end{itemize}
\par
\vspace{7px}


{\bf \large Machine Learning} 
\hfill \textit{May 2019}
\\ \textit{by Stanford University} on \emph{Coursera}
% \\ \emph{Coursera Certificate ID: BWS5UB3J7YTK}
\par\vspace{7px}


{\bf \large Cloud Computing Certification}
\hfill \textit{Aug - Sep 2018}
\\ \textit{by IIT Kharagpur} on \emph{NPTEL}
% \\ \emph{NPTEL Credential ID: NPTEL18CS44S21300023}
\par\vspace{7px}

% {\bf  \large Trainee}
% \hfill \textit{June 2016}
% \\ \textit{Guwahati Refinery Learning and Development Centre}



\end{rSection}
%    COURSES SECTION
%----------------------------------------------------------------------------------------
\end{document}
